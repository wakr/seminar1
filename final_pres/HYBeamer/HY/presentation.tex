% rubber: module pdftex
% rubber: path HY

%%%%%%%%%%%%%%%%%%%%%%%%%%%%%%%%%%%%%%%%%%%%%%%%%%%%%%%%%%
%  Usage HOWTO                                           %
%  -----------                                           %
%  * Compilation:          rubber presentation           %
%  * Cleaning:             rubber --clean presentation   %
%  * Force recompilation:  rubber --force presentation   %
%%%%%%%%%%%%%%%%%%%%%%%%%%%%%%%%%%%%%%%%%%%%%%%%%%%%%%%%%%

\documentclass[t,12pt,pdftex]{beamer}
\usepackage{helvet}
\usepackage{times}
\usepackage{courier}

\usepackage[T1]{fontenc}
\usepackage[english]{babel}

\usepackage{amssymb}
\usepackage{amsmath}
\usepackage{amsfonts}
\usepackage{graphicx}
\usepackage{color}
\usepackage{url}
\usepackage{textpos}
\usepackage{xspace}
\usepackage{array}

\graphicspath{{./fig/}}

% theme options: hy/ml/hum, rovio/sinetti, hiit
% default: hy,rovio

%\usetheme[hy]{HY}
%\usetheme[hy,sinetti]{HY}
%\usetheme[hum,rovio]{HY}
\usetheme[ml,rovio]{HY}
%\usetheme[ml,rovio,hiit]{HY}


\title{Tools for plagiarism detection}
%\author{Name Name \scriptsize \guilsinglleft{}name.name@helsinki.fi\guilsinglright{}}
\author{Kristian Wahlroos}
\institute{University of Helsinki\\Department of Computer Science}
\date{\today}

\begin{document}

%\selectlanguage{english}

\HyTitle

\begin{frame}
	\frametitle{Outline}
	\tableofcontents
\end{frame}


%\AtBeginSection[]
%{
%  \begin{frame}<beamer>
%    \frametitle{Outline}
%    \tableofcontents[currentsection]
%  \end{frame}
%}

\section{Introduction}

\subsection{Problem}

\begin{frame}
	\frametitle{Introduction}
	
	\begin{itemize}
		\item MOOC's have gained popularity in recent years
		\begin{itemize}
			\item Especially programming related MOOC's\footnote{\url{http://blog.edx.org/10-most-popular-courses-edx-courses-in-2016}}
			\item Independent assignments
			\item No live-presence required
		\end{itemize}
		\item Number of students often large
		\item Trust is thus usually one-sided
		\begin{itemize}
			\item Belief that students do tasks by themselves
			\item Not actively monitored
			\item Cheating is in form of plagiarism
			\item Many potential plagiarism scenarios
		\end{itemize} 
	\end{itemize}
\end{frame}

\subsection{Plagiarism}


\begin{frame}
	\vspace{1in}
	\begin{itemize}
			\item Source code plagiarism is a problem consisting many forms
		\begin{itemize}
			\item Straight plagiarism
			\item Too intense group work 
			\item Code sharing
			\item Obfuscation
		\end{itemize}
		\item Lots of students $\rightarrow$ impossible to detect manually in reasonable time
		\begin{itemize}
			\item Lot of data available
			\item Need for automated tools
		\end{itemize}
	\end{itemize}
\end{frame}

\subsection{Motivation}

\begin{frame}
	\frametitle{In this study}
	\begin{itemize}
		\item Finding a suitable machine learning tool set for detecting source code plagiarism
		\item Motivated by
		\begin{itemize}
			\item Could be used in University of Helsinki's course \textit{Introduction to programming}
			\item Interesting topic
			\item Machine learning methods benefit from a lot of data
		\end{itemize}
		\item Results reflected to the usage in a academic course
	\end{itemize}
\end{frame}




\section{Methodology}
\subsection{Literature review}


\begin{frame}
	\frametitle{Methodology}
	\begin{itemize}
		\item Performing literature review with \textit{Google Scholar}
		\item Collected 8 papers 
		\item Two-step search process
		\begin{itemize}
			\item Limit by overall keywords occurrences
			\item Limit by title/abstract/keywords
		\end{itemize}
		\item Keywords
		\begin{itemize}
			\item Direct matches: \textbf{machine learning, plagiarism, code, programming}
			\item Non-direct: \textbf{authorship, identification} 
		\end{itemize}
	\end{itemize}
\end{frame}

\begin{frame}
	\vspace{1in}
	\begin{itemize}
		\item Limited years starting from 2006
		\begin{itemize}
			\item Believed to contain more recent programming languages
			\item MOOC's are relatively new concept
			\item Machine learning methods have changed
		\end{itemize}
		\item Doing comparison between papers
		\begin{itemize}
			\item Model accuracy
			\item Data
			\item Machine learning methodology
			\item Feature extraction
		\end{itemize}
	\end{itemize}
\end{frame}

\section{Results}
\subsection{Overview}
\begin{frame}
	\frametitle{Results}
	\begin{itemize}
		\item 8 papers from 2007 to 2015
		\begin{itemize}
			\item[1)] \textit{A machine learning based tool
for source code plagiarism detection}, 2011
			\item[2)] \textit{De-anonymizing
programmers via code stylometry}, 2015
			\item[3)] \textit{Detecting outsourced student
programming assignments}, 2008
			\item[4)] \textit{Pde4java: Plagiarism detection
engine for java source code: a clustering approach}, 2008
		\end{itemize}
	\end{itemize}
\end{frame}

\begin{frame}
	\vspace{0.5in}
	\begin{itemize}
	\item[5)] \textit{A probabilistic approach to source code authorship identification}, 2007
			\item[6)] \textit{Using code metric histograms and genetic algorithms to perform author identification for software forensics}, 2007
			\item[7)] \textit{Who wrote
this code? identifying the authors of program binaries}, 2011
			\item[8)] \textit{An application for plagiarized source code detection based on a parse
tree kernel}, 2013
	\end{itemize}
\end{frame}

\begin{frame}
	\vspace{0.5in}
	\begin{itemize}
		\item Studies divide into two categories
		\begin{itemize}
			\item Attribute counting
			\item Structure based
		\end{itemize}
		\item Model accuracies are reported in two ways
		\begin{itemize}
			\item Traditional classification accuracy
			\item How close the model was to human labeling
		\end{itemize}
		\item Accuracies ranged from 69\% to over 90\%
		\begin{itemize}
			\item Highest used mixture of stylistic and structural approach
			\item E.g. 93\% same results compared to human validator
		\end{itemize}
	\end{itemize}
\end{frame}

\begin{frame}
	\vspace{0.5in}
	\begin{itemize}
		\item Plagiarism detection is close to authorship identification
			\begin{itemize}
			\item Classifying anonymous source code
			\item Clustering similar documents together
			\item Finding stylistic nuances
			\item Trying to capture the logical structure
		\end{itemize}
	\end{itemize}
\end{frame}

\subsection{Data}
\subsection{Methodologies}
\subsection{Feature extraction}

\section{Discussion}

\begin{frame}
	\frametitle{Yet another slide}
	
	content
\end{frame}

\end{document}
